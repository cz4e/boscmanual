\chapter{这是什么}
这是个人使用的用于编写手册的LaTex模版。

\section{在线使用模版}
可以使用texpage和overleaf在线导入(https://github.com/cz4e/boscmanual)或者手动导入,编译器选择XeLaTex。

\section{本地安装使用}
DYOR,编译器选择XeLaTex。

\section{模版文件介绍}
\begin{itemize}
    \item boscmaunal.cls: 模版类型文件,文件包含特定模版类型的格式和模版规范。
    \item main.tex: 主文件。
    \item logo: 模版logo目录。
    \item fonts: 模版字体目录。
    \item preface: 前言目录。
    \item chapter: 章节目录,其中包括用于章节主文件、引用文件、源代码目录和图片目录。
\end{itemize}

\section{模版编写简要教程}
手册每一章拥有独立目录(chapter1,chapter2,...),每一章的内容写在main.tex,引用写在references.bib中。编写完成后需要在模版主文件中添加对应章节的主文件。例如在includeonly中添加charpter1/main,并在对应位置添加include\{chapter1/main\}

% chapter reference
\bibliographystyle{IEEEtran}
\bibliography{chapter1/references}
